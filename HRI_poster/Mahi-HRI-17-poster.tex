\documentclass[portrait,a1paper,fontscale=0.4]{baposter} % Adjust the font scale/size here

\usepackage{graphicx} % Required for including images
\graphicspath{{figures/}} % Directory in which figures are stored

\usepackage{amsmath} % For typesetting math
\usepackage{amssymb} % Adds new symbols to be used in math mode

%\usepackage{subfigure,float}
%\usepackage{subfig}
%\usepackage{subfloat}
%\usepackage{subcaption}
%\usepackage{float}
\usepackage{pgf}
%\usepackage{minipage}
\usepackage{tikz}
\usetikzlibrary{shapes.geometric, arrows,backgrounds}
\usetikzlibrary{positioning,fit,calc}
\usepackage{scalefnt}
\usepackage{graphicx}
\usepackage[labelfont+=small]{subcaption}
\usepackage[linesnumbered,ruled,vlined]{algorithm2e}
\usepackage{booktabs} % Top and bottom rules for tables
\usepackage{enumitem} % Used to reduce itemize/enumerate spacing
\usepackage{palatino} % Use the Palatino font
\usepackage[font=small,labelfont=bf]{caption} % Required for specifying captions to tables and figures

\usepackage{multicol} % Required for multiple columns
\setlength{\columnsep}{1.5em} % Slightly increase the space between columns
\setlength{\columnseprule}{0mm} % No horizontal rule between columns

\usepackage{tikz} % Required for flow chart
\usetikzlibrary{shapes,arrows} % Tikz libraries required for the flow chart in the template

\newcommand{\compresslist}{ % Define a command to reduce spacing within itemize/enumerate environments, this is used right after \begin{itemize} or \begin{itemize}
\setlength{\itemsep}{1pt}
\setlength{\parskip}{0pt}
\setlength{\parsep}{0pt}
}

\definecolor{lightblue}{rgb}{0.145,0.6666,1} % Defines the color used for content box headers

\begin{document}

\begin{poster}
{
headerborder=closed, % Adds a border around the header of content boxes
colspacing=1em, % Column spacing
bgColorOne=white, % Background color for the gradient on the left side of the poster
bgColorTwo=white, % Background color for the gradient on the right side of the poster
borderColor=orange, % Border color
headerColorOne=orange, % Background color for the header in the content boxes (left side)
headerColorTwo=orange, % Background color for the header in the content boxes (right side)
headerFontColor=white, % Text color for the header text in the content boxes
boxColorOne=white, % Background color of the content boxes
textborder=roundedleft, % Format of the border around content boxes, can be: none, bars, coils, triangles, rectangle, rounded, roundedsmall, roundedright or faded
eyecatcher=true, % Set to false for ignoring the left logo in the title and move the title left
headerheight=0.1\textheight, % Height of the header
headershape=roundedright, % Specify the rounded corner in the content box headers, can be: rectangle, small-rounded, roundedright, roundedleft or rounded
headerfont=\Large\bf\textsc, % Large, bold and sans serif font in the headers of content boxes
%textfont={\setlength{\parindent}{1.5em}}, % Uncomment for paragraph indentation
linewidth=2pt % Width of the border lines around content boxes
}
%----------------------------------------------------------------------------------------
%	TITLE SECTION 
%----------------------------------------------------------------------------------------
%
{\includegraphics[height=4em]{OSUlogo}} % First university/lab logo on the left
{\bf\textsc{Learning to assess the cognitive capacity of human partners}} % Poster title\vspace{0.5em}
{\textsc{S. M. al Mahi, Matthew Atkins and Christopher Crick\\Oklahoma State University\hspace{12pt} }} % Author names and institution
% {\includegraphics[height=3em]{tamu-logo}} % Second university/lab logo on the right

%----------------------------------------------------------------------------------------
%	OBJECTIVES
%----------------------------------------------------------------------------------------

\headerbox{Objectives}{name=objectives,column=0,row=0}{


Enable robots to:
\begin{itemize}\compresslist
	
\item Learn to assess cognitive capacity of a human partner.
\item Choose their level of autonomy based on this cognitive assessment.
\item Reduce the human decision burden.
\item Improve task performance in difficult, chaotic and dynamic circumstances.

\end{itemize}

%\vspace{0.3em} % When there are two boxes, some whitespace may need to be added if the one on the right has more content
}


\headerbox{Overview of the Model}{name=overview,column=1,span=2,row=0}{
\centering
\begin{multicols}{2}
  \begin{minipage}{.52\textwidth}
  	\pgfdeclareimage[height=5.1cm, width =\textwidth]{overview}{./figures/stress_model}
  	\pgfuseimage{overview}
  \end{minipage}
  
  \begin{minipage}{.4\textwidth}
       \pgfdeclareimage[height= 4 cm, width = 5.5cm]{maze}{./figures/robots-in-maze}
       \begin{minipage}{0.35\textwidth}
	 	\centering
	 	\pgfuseimage{maze} %
       \end{minipage}\vspace{1em}
       \captionof{figure}{Experimental setup of maze game experiment for training model to predict cognitive stress}
   \end{minipage}

\end{multicols}
}

\headerbox{Motivations}{name=motivations,column=0,span=1,below=objectives}{ 
	\textbf{Motivation:}
    \begin{itemize}
    \item Inherent communication barriers between human and robot
    \item Controlling multiple robots becomes impossible: cognitive
      load, heterogeneous robots
    \item Complete automation impossible: new task environment
    \item Robots need to assess cognitive capacity of human-robot team
      for mutual benefits\cite{crandall2005validating}
    \end{itemize}}
\headerbox{Contributions}{name=contributions,column=0,span=1,below=motivations}{ 
    \textbf{Contributions:}
    We produce a robot that can:
    \begin{itemize}
    \item Compute a success metric.
    \item Connect observed human behavior to the task success.
    \item Learned model of human behavior is still useful in tasks
      when the robot can no longer self-evaluate.
 
    \end{itemize}
}

\headerbox{Physiological metrics}{name=trivial,column=0,below=contributions}{
  \begin{minipage}{\textwidth}
  \pgfdeclareimage[height=2 cm, width = 6.5 cm]{bioharness}{./figures/bioharness}
  \pgfuseimage{bioharness}
  Traditional metrics\cite{olsen2003metrics} for behavioral indicators (i.e. ECG,EEG) have drawbacks:
  \begin{itemize}\compresslist
  	\item Difficult to deploy in real-world task environments.
  	\item Instruments measure something different from task performance (correlation not strong, as seen below).
  	\item Still useful as baseline to evaluate other approaches.
  \end{itemize}  
  \end{minipage}
}

\headerbox{Feature Metrics}{name=contributions,column=0,below=trivial}{
%  \vspace{0.5em}
  \begin{minipage}{\textwidth}
    $E=[e_1,\dots,e_n]$ are measurable environmental features of task success.\\
    $H=[h_1,\dots,h_n]$ are human behavioral metrics which are ecologically valid for a navigation task.
    \pgfdeclareimage[width=6.5 cm]{metrics_E_H}{./figures/metrics_E_H}
    \centering
    \pgfuseimage{metrics_E_H}
  \end{minipage}
}

\headerbox{Experiments and Results }{name=methodology,column=1,span=2,below=overview}{
  \begin{multicols}{2}
  \begin{minipage}{.5\textwidth}
  \pgfdeclareimage[width = 5.5 cm]{coin}{./figures/coin_game_rviz}
  Our experiments consisted of two games, maze navigation\cite{crick2011human} and coin collection.
  \\\textbf{Maze game:}
  \begin{itemize}
  	\item The task in this game is to complete a maze (Fig.1) by controlling one or more Turtlebot robots
  	\item 2 min. long; collisions with walls are negatively rewarded
  	\item Complexity increases over time.
  	\item The maze game was used to collect the metrics in $E$ and calculate the success score $s$
  	\item The underlying function was modeled using $E$ and $s$ by
          using a random forest learning algorithm
  \end{itemize}
  \centering
  \pgfuseimage{coin}
  \captionof{figure}{Interface to human operator for coin game.}
  \end{minipage}

  % Coin Game
  \begin{minipage}{.48\textwidth}
  \pgfdeclareimage[width = 5.5 cm]{res1}{./figures/BoxWiskerTimesCompMaualVsAuto2}
  \textbf{Coin Game:}
  \begin{itemize}
  	\item The goal is to collect 5 coins instead of completing the maze
  	\item The timeout for completing the next goal is decremented after each coin collection
  	\item Human behavioral metrics $H$ were evaluated by the learned model to estimate stress $\hat{s}$
  	\item Under one test condition, robot navigated autonomously upon detecting high cognitive stress
  \end{itemize}
  \centering \pgfuseimage{res1} \captionof{figure}{\textbf{Learned
      model's contribution to task success:} Coin game task penalties
    in manual vs. autonomous assistance modes across 34 test
    subjects. $p < 0.05$ in both instances.}
  \end{minipage}
  \end{multicols}
}

\headerbox{Prediction vs Evidence}{name=demos,column=1,below=methodology}{ 
  \pgfdeclareimage[height= 4 cm, width = 5.5 cm]{res2}{./figures/prediction_vs_b_p_2.png}
  \textbf{Result:}
  The robot correctly predicts an operator's cognitive load in Figure 4.
  \centering
  \pgfuseimage{res2}
  \captionof{figure}{Robot's estimated cognitive stress level modestly correlates
  with physiological metrics (breathing rate measured with a bioharness).}
}

\headerbox{References}{name=references,column=2,below=methodology}{
  
  \renewcommand{\section}[2]{\vskip 0.05em} 
  \nocite{*} 
  \small{
    \bibliographystyle{unsrt}
    \bibliography{posterbib}
    
  }
}


\end{poster}

\end{document}
